%%%%%%%%%%%%%
% 
% Alexander Powell
% Finite Automata
% Homework Assignment #4
% 10.06.2015
% 
%%%%%%%%%%%%%

\documentclass[11pt]{article}

\usepackage{times,mathptm}
\usepackage{pifont}
\usepackage{exscale}
\usepackage{latexsym}
\usepackage{amsmath}
\usepackage{amssymb}
\usepackage{amsthm}
\usepackage{epsfig}
\usepackage{tikz}


\textwidth 6.5in
\textheight 9in
\oddsidemargin -0.0in
\topmargin -0.0in

\parindent 0pt     % How much the first word of a paragraph is indented. 
\parskip 0pt	   % How much extra space to leave between paragraphs.

\begin{document}

\begin{center}             % If you only centering 1 line use \centerline{}
\begin{LARGE}
{\bf Finite Automata Homework 4}
\end{LARGE}
\vskip 0.25cm      % vertical skip (0.25 cm)

Due: Tuesday, Oct 6\\  % force new line
Alexander Powell
\end{center}

\begin{enumerate}

\item
Prove by the pumping lemma that the language $A$ of strings of $0$s and $1$s whose length is a power of two is not regular.

\begin{proof}

Let's begin by assuming that $A$ is a regular language so that we can apply the pumping lemma to $A$.  If we let $P$ be the pumping length we can choose our value of $s = 1^P$.  Clearly we have that $|1^P| \geq P$, so it satisfies the length requirement.  We can also express $|s| = |x| + |y| + |z| = 2^n, n \geq 0$.  

From here we have two cases:
\begin{enumerate}

\item
$|s|=|y|$ (or $x = \epsilon$)

In this case, we know that $|y| = 2^n$.  By choosing $i = 3$ then $|y^3| = |3 \times 2^n|$ and $3 \times 2^n$ is clearly not a power of $2$.  This is a contradiction to our assumption.  

\item
$|s|>|y|$ (or $x \neq \epsilon$)

In the cases where $x$ is not the empty string then $|xy^2z| = |xyz| + |y|$.  This can be rewritten to equal $2^n + |y|$.  Because the next power of $2$ after $2^n$ is $2^n + 2^n$ and $|y| < 2^n$ then $2^n + |y| < 2^n + 2^n$ so it cannot be a power of $2$.  Again, this gives us a contradiction to our assumption.  

Because we reached a contradiction to the pumping lemma in both cases after assuming that $A$ is regular, we can conclude that the language $A$ is not regular.  

\end{enumerate}

\end{proof}

\item
Are the following languages regular? Prove you answers. 
\begin{enumerate}

\item
$A = \{ uww^Rv | u,v,w \in {0, 1}^+ \}$

\begin{proof}
To prove this is a regular language we simply need to come up with a DFA, NFA, or regular expression that will generate the language.  The following regular expression generates every member of the language $A$.  

$$ (0 \cup 1)^+(00 \cup 11)^+(0 \cup 1)^+ $$

Therefore, this proves that $A$ is regular.  
\end{proof}

\item
$B = \{ uww^Rv | u,v,w \in {0, 1}^+, |u| \geq |v| \}$

The language $B$ is not a regular language.  We use a proof by contradiction to establish this.  

\begin{proof}

Let's assume to the contrary that $B$ is a regular language.  Then the pumping lemma applies to $B$.  This leads to two different cases: the pumping length $P$ is either even or odd.  

\begin{enumerate}

\item Case $1$: $P$ is even

We take our $s$ to be $s = (01)^{P/2}00(10)^{P/2}$ so we can deduce the length of $s$ is $|s| = 2p+2$.  In this case, $w$ goes up through the first $0$ in the $00$ and $w^R$ begins at the second $0$ in $00$.  In this case we know that $u$ and $v$ should be of equal length.  When we examine the string resulting from fixing $i=0$ we know $|y|$ has to be at least $1$ and therefore $y^0$ will put one more character after $w^R$ than before $w$.  This is a contradiction to the rule that $|u| = |v|$.  

\item Case $2$: $P$ is odd

In the second case, our choice of $s$ can be written as $s = (01)^{(p-1)/2}100(01)^{(p-1)/2}$ and again $|s| = 2p+2$.  Similarly to the first case, when we fix $i=0$ we get another contradiction to the rule stating $u| \geq |v|$.  

Therefore, after seeing in both cases that we arrive at a contradiction to our originial assumption, we can conclude that $B$ is not a regular language.  

\end{enumerate}

\end{proof}

\end{enumerate}

Collaborators: Derek O'Connell

\item
Problem 1.54 (a) on page 91. Use closure properties.

We are given the language $F = \{ a^i b^j c^k | i,j,k \geq 0$ and if $i = 1$, then $j = k \}$.  
Prove that $F$ is not regular:

\begin{proof}

To prove $F$ is not regular, let's assume to the contrary that $F$ is regular and thus the pumping lemma applies.  Now, if we come up with a new language $L = \{ a b^i c^j | i,j \geq 0 \}$, then $L$ is clearly regular because it can be described by the regular expression $a b^* c^*$ and therefore the intersection of these two languages which we can call $G = F \cap L$ is regular as well, according to the closure properties of regular languages.  The intersection, $G$, of these languages can further be written as $G = \{ a b^i c^i | i \geq 0 \}$.  Now, using the pumping lemma with $P$ being the pumping length we have $s = ab^Pc^P \in G$ and $|s| > P$ so there exists $x,y,$ and $z$ such that $xy^iz \in G, \forall i \geq 0$ and $|y| > 0$ and $|xy| \leq P$.  Now, let's fix $i=2$ and from here there are two cases to consider:

\begin{enumerate}
\item
Case $1$: $y$ includes $a$ (so $x=\epsilon$)

If $y$ includes $a$ then our $s$ looks like $xyyz$ which contains two $a$s, giving us a contradiction.  

\item
Case $2$: $y$ does not include $a$ (so $x \neq \epsilon$)

If $y$ does not include $a$ then it either contains both $b$ and $c$ or either $b$ or $c$.  In these two subcases then there are either at least $2$ $bc$s in $xyyz$ or the number of $b$s and $c$s in $xyyz$ is not equal, respectively.  Again, this contradicts our assumtion.  

In all cases we eventually arrive at a contradiction to the statement that $G = F \cap L$ was regular.  Since we know $L$ is definitely a regular language then we can conclude that $F$ is not regular.  

\end{enumerate}

\end{proof}

\end{enumerate}

\end{document}









